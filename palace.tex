\documentclass{article}
\title{Kafka's The Palace}
\author{Shad Gregory}
\date{}
\begin{document}
\maketitle
\begin{verse}
  \begin{center}
    \textbf{1} \\
  \end{center}
  When K. arrived, it was late in the night, \\
  The village was covered under deep snow, \\
  And the castle hill was nowhere in sight. \\
  Fog and darkness enveloped him and though \\
  The sky's faintest glow of light could not show \\
  The outline of the great Palace, K. stood \\
  On the wooden bridge, the darkness below, \\
  The seeming emptiness was understood \\
  To offer our hero the start of something good.
\end{verse}
\begin{verse}
  \begin{center}
    \textbf{2} \\
  \end{center}
  And then, a decisive intake of air, \\
  K. descended to the village below \\
  Hoping to find lodgings anywhere there. \\
  In the inn, peasants still stirred to and fro \\
  In spite of the hour. There were no rooms, though \\
  The Innkeeper offered a sack of straw, \\
  And K.'s weariness had brought him so low \\
  That he eagerly accepted as fair \\
  The innkeeper's offer to sleep on the floor there. \\
\end{verse}
\begin{verse}
  \begin{center}
    \textbf{3} \\
  \end{center}
  But soon then K. was roused from his slumber \\
  By a young man with a thespian's face. \\
  The peasants were still there too in number \\
  And many had turned from their beer in case \\
  An entertaining spectacle took place. \\
  The young man was dressed in fancy city \\
  Clothes; his eyes were narrowed; it seemed the case \\
  The young man was the son of a pretty \\
  Big deal, and he was not trying to be witty!
\end{verse}
\newpage
\begin{verse}
    \begin{center}
    \textbf{4} \\
  \end{center}
A big deal indeed, the son of the Palace \\
Steward stood over K., his eyebrows strong, \\
Ready to torment with polite malice \\
Our poor K. over his excursion along \\
The village's outskirts lurking among \\
The shadows and darkness of the late hour. \\
You have entered the village was his song, \\
And the right to stay in any bower \\
Or hut, resided only in Count Westwest's power.
\end{verse}
\begin{verse}
    \begin{center}
    \textbf{5} \\
  \end{center}
Half sitting up and straightening his hair \\
K. nonchalantly glanced up at the crowd;  \\
The innkeeper, the peasants in their chairs,  \\
The young man asking if K. should be allow'd, \\
All waiting there for K. to speak out loud \\
His intentions on such a snowy night. \\
"Where am I?" K. asked as if he were proud \\
of his ignorance,  "I'm lost in the night." \\
He cried, "Castle you say? But there was not one in sight."
\end{verse}
\begin{verse}
  \begin{center}
    \textbf{6} \\
  \end{center}
  The young man was astonished by K.'s act, \\
  "Why indeed, the Castle of Count Westwest!" \\
  "And you need the Count's permission, in fact, \\
  For a weary travellor to simply rest \\
  Overnight?" asked K. upon being press'd \\
  By the expecting crowd. Was it a dream \\
  That gave to him the notion that a guest \\
  Could be so cruelly turned out? It did seem \\
  To beggar belief. Such cruelty K. could not gleam.
\end{verse}
\begin{verse}
  \begin{center}
    \textbf{7} \\
  \end{center}
  "You must have permission!" was the reply. \\
  And with that, the dramatic young man turn'd \\
  To his audience, and said with a sigh, \\
  "Or maybe it's not required to have earn'd \\
  The Count's blessings!" And now having so learn'd \\
  The conditions of discretely dwelling \\
  overnight, grasping that which so concern'd \\
  the crowd, K. yawned, and perhaps overselling \\
  his nonchalance, announced his plans without yelling.
\end{verse}
\newpage
\begin{verse}
  \begin{center}
    \textbf{8} \\
  \end{center}
  "Now, if it is permission that I need," \\
  Said K. "Then it is permission I seek." \\
  And as if he were about to proceed, \\
  Cast off his blanket with nary a peek \\
  At the shocked crowd, barely able to speak. \\
  "Permission from whom?" sputtered the young man, \\
  "At this midnight hour?" he said with a shreik. \\
  "It isn't possible?" and K. began \\
  To yawn and stretch. "See, I like to sleep when I can!"
\end{verse}
\begin{verse}
  \begin{center}
    \textbf{9} \\
  \end{center}
  The young man was beside himself with rage, \\
  "Why you're not but a low-down dirty bum!" \\
  With a passion found only on the stage. \\
  "The count demands respect! Not some sass from \\
  A common tramp who's lower than pond scum! \\
  You must depart the count's territory \\
  At once!" At this, K. was able to drum \\
  Up the peace of a saint in God's glory, \\
  "Enough!" he said, and K. then began his story.
\end{verse}
\begin{verse}
  \begin{center}
    \textbf{10} \\
  \end{center}
  Does K. feel despair? Does he cry in the night? \\
  Is he so fixated on his mission \\
  That he no longer dreads the morning light? \\
  Why has he come here without permission? \\
  Travelled so far on this expedition \\
  Without a companion to help him through \\
  The snow and darkness with precision. \\
  Where is his family? Are they so few \\
  That K. was attracted to the castle in view?
\end{verse}
\begin{verse}
  \begin{center}
    \textbf{11} \\
  \end{center}
  "I've had enough of your nonsense." said K, \\
  "The Innkeeper and these good gentlemen \\
  Are my witnesses should I need to sway \\
  A jury of my peers. I take it then \\
  You would like to know why I am here in \\
  Your village. I am the land surveyor \\
  Sent for by the Count. Now there, you see when \\
  I saw the snow, layer upon layer, \\
  I sat out on the trek after a hopeful prayer.
\end{verse}
\newpage
\begin{verse}
  \begin{center}
    \textbf{12} \\
  \end{center}
  ``But, unfortunately, I lost my way \\
  More than a few times and arrived so late \\
  That I knew it was too late in the day \\
  To report to the Castle in my state. \\
  This is why I chose to accept my fate \\
  And make do with camping out on the floor \\
  Here in the corner as much as I hate \\
  To give up the comforts of a locked door \\
  And a sweet bed, I knew my sleep would not be poor.''
\end{verse}
\begin{verse}
  \begin{center}
    \textbf{13} \\
  \end{center}
  "Tommorrow my assistants will arrive \\
  Via carriage with the equipment in tow. \\
  Now that's all that I'm willing to contrive \\
  As far as an explanation will go. \\
  Now goodnight fellas and, please, go pound snow!" \\
  K. turned to the stove and pulled his blanket tight. \\
  The Inn's mob retreated after K.'s show, \\
  Confused by this information's new light, \\
  They conversed in hushed tones while keeping K.in sight.
\end{verse}
\begin{verse}
  \begin{center}
    \textbf{14} \\
  \end{center}
  "Surveyor?" the word was tossed back and forth, \\
  And then a silence fell over the mob. \\
  The young man, eager to show off his worth, \\
  And now determined to finish the job, \\
  Whispered in a tone so as not to rob \\
  K. of his sleep but loud enough to hear \\
  "I'll call the Castle, ask about this slob, \\
  And check his story." he said with a sneer. \\
  He headed to the phone and brought it close to his ear.
\end{verse}
\begin{verse}
  \begin{center}
    \textbf{15} \\
  \end{center}
  "Good Goddamn!" thought K. to himself, "This place \\
  Is decked out to the nines! They have a phone?" \\
  Said telephone was crowded in a space \\
  Directly above K.'s head. In his own \\
  Weariness, among them all, he was alone \\
  In overlooking the infernal thing. \\
  Now K.'s restful sleep was sure to be blown \\
  By the eager fellow's attempt to ring \\
  The Palace. And now poor K. had to hear him sing.
\end{verse}
\newpage
\begin{verse}
  \begin{center}
    \textbf{16} \\
  \end{center}
  Then the question was, would K. allow it? \\
  He decided to allow it, but now \\
  It was the case he could find no merit \\
  In feigning sleep, he flipped o'er with a scow \\
  And waited for the young man to find how \\
  To inquire without disrupting K.'s sleep. \\
  Across the way the dim light did allow \\
  K. to see the bauren together deep \\
  In discussion and tightly piled in a heap.
\end{verse}
\begin{verse}
  \begin{center}
    \textbf{17} \\
  \end{center}
  K.'s arrival was no trivial news. \\
  Surveyors don't pop up every day! \\
  Every landlord had something to lose \\
  If the Count changed the lines any old way. \\
  The kitchen door was opened all the way, \\
  And it's frame filled by the landlady's form. \\
  The host, eager to report on the fray, \\
  Tiptoed in her direction to inform \\
  The mighty Landlady of the incoming storm.
\end{verse}
\begin{verse}
  \begin{center}
    \textbf{18} \\
  \end{center}
  The telephone conversation began. \\
  The Palace Governor was sound asleep, \\
  But one of his lackeys was the night man, \\
  A certain Herr Fritz, who was known to keep \\
  Some abysmally late hours sometimes deep \\
  Into the night, was awake. The young man, \\
  Going by Schwarzer, proceeded to leap \\
  Into how he had found K., worn and wan, \\
  Sleeping on a dirty straw sack, so he began.
\end{verse}
\end{document}
